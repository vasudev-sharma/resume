%-------------------------
% Resume in Latex
% Author : Jake Gutierrez
% Based off of: https://github.com/sb2nov/resume
% License : MIT
%------------------------

\documentclass[letterpaper,11pt]{article}

\usepackage{latexsym}
\usepackage[empty]{fullpage}
\usepackage{titlesec}
\usepackage{marvosym}
\usepackage[usenames,dvipsnames]{color}
\usepackage{verbatim}
\usepackage{enumitem}
\usepackage{url}
\usepackage{hyperref}
\usepackage{fancyhdr}
\usepackage[english]{babel}
\usepackage{tabularx}
\usepackage{fontawesome5}
\usepackage{multicol}
\setlength{\multicolsep}{-3.0pt}
\setlength{\columnsep}{-1pt}
\input{glyphtounicode}
\usepackage{verbatim}


%----------FONT OPTIONS----------
% sans-serif
% \usepackage[sfdefault]{FiraSans}
% \usepackage[sfdefault]{roboto}
% \usepackage[sfdefault]{noto-sans}
% \usepackage[default]{sourcesanspro}

% serif
% \usepackage{CormorantGaramond}
% \usepackage{charter}


\pagestyle{fancy}
\fancyhf{} % clear all header and footer fields
\fancyfoot{}
\renewcommand{\headrulewidth}{0pt}
\renewcommand{\footrulewidth}{0pt}

% Adjust margins
\addtolength{\oddsidemargin}{-0.6in}
\addtolength{\evensidemargin}{-0.5in}
\addtolength{\textwidth}{1.19in}
\addtolength{\topmargin}{-.7in}
\addtolength{\textheight}{1.4in}

\urlstyle{same}

\raggedbottom
\raggedright
\setlength{\tabcolsep}{0in}

% Sections formatting
\titleformat{\section}{
  \vspace{-4pt}\scshape\raggedright\large\bfseries
}{}{0em}{}[\color{black}\titlerule \vspace{-5pt}]

% Ensure that generate pdf is machine readable/ATS parsable
\pdfgentounicode=1

%-------------------------
% Custom commands
\newcommand{\resumeItem}[1]{
  \item\small{
    {#1 \vspace{-2pt}}
  }
}

\newcommand{\classesList}[4]{
    \item\small{
        {#1 #2 #3 #4 \vspace{-2pt}}
  }
}

\newcommand{\resumeSubheading}[4]{
  \vspace{-2pt}\item
    \begin{tabular*}{1.0\textwidth}[t]{l@{\extracolsep{\fill}}r}
      \textbf{#1} & \textbf{\small #2} \\
      \textit{\small#3} & \textit{\small #4} \\
    \end{tabular*}\vspace{-7pt}
}

\newcommand{\resumeSubSubheading}[2]{
    \item
    \begin{tabular*}{0.97\textwidth}{l@{\extracolsep{\fill}}r}
      \textit{\small#1} & \textit{\small #2} \\
    \end{tabular*}\vspace{-7pt}
}

\newcommand{\resumeProjectHeading}[2]{
    \item
    \begin{tabular*}{1.001\textwidth}{l@{\extracolsep{\fill}}r}
      \small#1 & \textbf{\small #2}\\
    \end{tabular*}\vspace{-7pt}
}

\newcommand{\resumeSubItem}[1]{\resumeItem{#1}\vspace{-4pt}}

\renewcommand\labelitemi{$\vcenter{\hbox{\tiny$\bullet$}}$}
\renewcommand\labelitemii{$\vcenter{\hbox{\tiny$\bullet$}}$}

\newcommand{\resumeSubHeadingListStart}{\begin{itemize}[leftmargin=0.0in, label={}]}
\newcommand{\resumeSubHeadingListEnd}{\end{itemize}}
\newcommand{\resumeItemListStart}{\begin{itemize}}
\newcommand{\resumeItemListEnd}{\end{itemize}\vspace{-5pt}}

%-------------------------------------------
%%%%%%  RESUME STARTS HERE  %%%%%%%%%%%%%%%%%%%%%%%%%%%%
% 

\begin{document}



\begin{center}
    {\Huge \scshape Vasudev Sharma} \\ \vspace{1pt}
    Toronto, Canada \\ \vspace{1pt}
    \small \raisebox{-0.1\height}\faPhone\ 647-533-4447 ~ \href{mailto:vasu@cs.toronto.edu}{\raisebox{-0.2\height}\faEnvelope\  \underline{vasu@cs.toronto.edu}} ~ 
    \href{https://linkedin.com/in/vasudev-sharma-}{\raisebox{-0.2\height}\faLinkedin\ \underline{linkedin.com/in/vasudev-sharma-}}  ~
    \href{https://github.com/}{\raisebox{-0.2\height}\faGithub\ \underline{github.com/vasudev-sharma}}
    \vspace{-8pt}
\end{center}

% -----------SUMMARY --------------
\begin{comment}
    \section{Summary}
    An open source contributor with one 1-year experience as a Machine Learning Engineer
\end{comment}
%-----------EDUCATION-----------
\section{Education}
  \resumeSubHeadingListStart
    \resumeSubheading
      {University Of Toronto}{Sep. 2021 -- Present}
      {Master of Science in Applied Computing (Computer Science)}{Toronto, Canada}
    \resumeSubheading
        {VIT University}{Sep. 2016 - June 2020}
        {B.Tech in Computer Science}{Vellore, India}
  \resumeSubHeadingListEnd

%------RELEVANT COURSEWORK-------
\section{Relevant Coursework}
    %\resumeSubHeadingListStart
        \begin{multicols}{4}
            \begin{itemize}[itemsep=-10pt, parsep=15pt]
                \item\small Machine Learning (Audit)
                \item\small ML in Healthcare
                \item Information Visualization
                \item Computer Vision 
                \item Deep Learning
                \item \small Natural Language Computing (Audit)
            \end{itemize}
        \end{multicols}
        \vspace*{1.0\multicolsep}
    %\resumeSubHeadingListEnd

%-----------EXPERIENCE-----------
\section{Experience}
  \resumeSubHeadingListStart

    \resumeSubheading
      {University of Toronto}{Sept 2021 -- Present}
      {Teaching Assistant}{Toronto, Canada}

        % \resumeSubHeadingListStart
        \resumeSubSubheading{\href{https://chanb.github.io/teaching}{ CSCC11:Introduction to Machine Learning }}{Winter 2022}{}{}
        \resumeSubSubheading{ \href{http://www.brianharrington.net/}{CSCA20: Introduction to programming }}{Fall 2021}{}{}
    %   \resumeItemListStart
      
    %         \resumeSubItem{Led tutorials on introductory undergraduate level course on Python}
    %         \resumeSubItem{Conducted office hours, graded exams and assignments}
    %     % \resumeSubHeadingListEnd
        
    %   \resumeItemListEnd

    \resumeSubheading
      {NeuroPoly, University of Montreal}{Nov 2020 -- August 2021}
      {Machine Learning Engineer}{Montreal, Quebec, Canada}
      \resumeItemListStart
        \resumeItem {Developed an open source software     \href{https://github.com/neuropoly/axondeepseg}{AxonDeepSeg} - Axon / Myelin segmentation using Deep Learning.}
         \resumeItem {Implemented and integrated U-Net model for segmentation on Keras framework for histological data ( SEM and TEM).}
        \resumeItem {Fine-tuned models resulting in a performance gain of \textbf{5\%}, refactored \textbf{40\% codebase} and performed an exhaustive comparative analysis with state-of-art methods.}
        \resumeItem {Researched and incorporated dynamic functionality for handling overlapping patch effect on microscopy images}
    \resumeItemListEnd

    \resumeSubheading
    {CNRS, CerCo lab}{Dec 2019 - Jun 2020}
    {Visiting Deep Learning Research Intern}{Toulouse, France}
        \resumeItemListStart
            \resumeItem{Researched the influence of EEG on stimulus, stimulus on EEG, and EEG on EEG primarily for the occipital electrodes.}\resumeItem{Improved correlation value(r) by \textbf{13\%} and improvised on the next \textbf{1 sec horizon time steps} in comparison to the baseline models using state-of-the-art time series models.}
            \resumeItem{Experimented the study; \textit{"In Alpha Oscillations strong perceptual echoes exist at 10Hz frequency"} with various architectures - 1D CNN, LSTM, WaveNet, Conv-LSTM, ARIMA, and an ensemble of these models.}
        \resumeItemListEnd
  \resumeSubHeadingListEnd
\vspace{-16pt}

%-----------PROJECTS-----------
\begin{comment}
\section{Projects}
    \vspace{-5pt}
    \resumeSubHeadingListStart
      \resumeProjectHeading
          {\textbf{Gym Reservation Bot} $|$ \emph{Python, Selenium, Google Cloud Console}}{January 2021}
          \resumeItemListStart
            \resumeItem{Developed an automatic bot using Python and Google Cloud Console to register myself for a timeslot at my school gym.}
            \resumeItem{Implemented Selenium to create an instance of Chrome in order to interact with the correct elements of the web page.}
            \resumeItem{Created a Linux virtual machine to run on Google Cloud so that the program is able to run everyday from the cloud.}
            \resumeItem{Used Cron to schedule the program to execute automatically at 11 AM every morning so a reservation is made for me.}
          \resumeItemListEnd
          \vspace{-13pt}
      \resumeProjectHeading
          {\textbf{Ticket Price Calculator App} $|$ \emph{Java, Android Studio}}{November 2020}
          \resumeItemListStart
            \resumeItem{Created an Android application using Java and Android Studio to calculate ticket prices for trips to museums in NYC.}
            \resumeItem{Processed user inputted information in the back-end of the app to return a subtotal price based on the tickets selected.}
            \resumeItem{Utilized the layout editor to create a UI for the application in order to allow different scenes to interact with each other.}
          \resumeItemListEnd 
          \vspace{-13pt}
          \resumeProjectHeading
          {\textbf{Transaction Management GUI} $|$ \emph{Java, Eclipse, JavaFX}}{October 2020}
          \resumeItemListStart
            \resumeItem{Designed a sample banking transaction system using Java to simulate the common functions of using a bank account.}
            \resumeItem{Used JavaFX to create a GUI that supports actions such as creating an account, deposit, withdraw, list all acounts, etc.}
            \resumeItem{Implemented object-oriented programming practices such as inheritance to create different account types and databases.}
          \resumeItemListEnd 
    \resumeSubHeadingListEnd
\vspace{-15pt}
\end{comment}


% -------------------- PUBLICATIONS ------------------
\section{Publications}
    \resumeSubHeadingListStart
        \resumeSubheading{AxonDeepSeg: Automatic Myelin and Axon Segmentation Using Deep Learning}{July 2020}{(\href{https://cdn-akamai.6connex.com/645/1827//ohbm2020_boudreau_FINAL_15921677822526191.pdf}{LINK})}{OHBM 2020, Canada}
 
        \resumeSubheading{High Dimensional Fuzzy Outlier Detection}{August 2019}{(\href{http://ajiips.com.au/papers/V16.1/v16n1_49-59.pdf}{LINK})}{ICONIP, Australia}


        \resumeSubheading{A Fuzzy Constraint Based Method for Outlier Detection}{August 2019}{(\href{https://link.springer.com/chapter/10.1007/978-3-030-26766-7_47}{LINK})}{ICIC2019, China}


    \resumeSubHeadingListEnd

%-----------PROGRAMMING SKILLS-----------
\section{Technical Skills}
 \begin{itemize}[leftmargin=0.15in, label={}]
    \small{\item{
     \textbf{Languages}{: Python, Shell Script, HTML} \\
     \textbf{Developer Tools}{: VS Code, Google Cloud Platform} \\
     \textbf{Technologies/Frameworks}{: PyTorch, NumPy, Scikit-learn, Pandas, Keras, OpenCV, Git, Docker, GitHub, AWS} \\
    }}
 \end{itemize}
 \vspace{-16pt}

%-----------INVOLVEMENT---------------
\section{Achievements / Awards}
    \resumeSubHeadingListStart
        \resumeSubheading{Vector Scholarship in Artificial Intelligence 2021}{September 2021}{Scholarship ({\href{https://vectorinstitute.ai/2021/05/10/newest-vector-ai-scholarship-recipients-join-growing-talent-pool-in-ontario/}{LINK}})}{Vector Institute and University of Toronto}
        \resumeSubheading{Charpak Lab France Scholarship}{September 2020}{Award and Scholarship ({\href{https://www.inde.campusfrance.org/result-of-the-charpak-lab-scholarship-programme-2019}{LINK}})}{Government of France}
        \resumeSubheading{Dean List of Academic Intelligence}{2016 -- 2020}{Award}{VIT University}
        \resumeSubheading{Special Achiever Award}{2019}{Award (\href{https://www2.slideshare.net/secret/9CIVAk7yffpwUx}{LINK})}{VIT University}
    \resumeSubHeadingListEnd



\end{document}